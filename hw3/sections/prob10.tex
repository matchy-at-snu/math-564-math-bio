\begin{homeworkProblem}[10]
\begin{enumerate}
    \item 
    If $m > 1$, $f(N_t, P_t) = e^{-\frac{1}{aP_t}}$, where $aP_t$ is raised to some power. As $P_t$ grows, $f(N_t, P_t)$ decreases, which has no biological significance.
    
    \item 
    \begin{align*}
        N_{t+1} &= \lambda N_{t} e^{-(aP_t)^{1-m}}\\
        P_{t+1} &= N_t (1 - e^{-(aP_t)^{1-m}})
    \end{align*}
    
    \item
    Since m represents a decrease in search efficiency for the parasites due to large density, introducing this parameter allows lower $q$ values, thus, more elevated stable states. Lower $q$ values also allow the system to be stable with a wide range of r values. 
    
    Steady State Analysis: \begin{align*}
        \bar{N} &= \lambda \bar{N} (\exp(a\bar{P}^{1-m}))\\
        \bar{P} &= \bar{N} (1 - \exp(-(a \bar{P})^{1-m}))
    \end{align*}
    Solving for $\bar{N}$ and $\bar{P}$, \begin{align*}
        \bar{P} &= \frac{ln(\lambda)^{\frac{1}{1-m}}}{a}\\
        \bar{N} &= \frac{\bar{P}}{1-\lambda}
    \end{align*}
    Taking the partial derivatives for stability analysis, \begin{align}
        a_{11} &= 1 \\
        a_{12} &= -\frac{\lambda^2}{\lambda - 1}ln(\lambda)(1-m) \\
        a_{21} &= \frac{\lambda - 1}{\lambda} \\
        a_{22} &= \frac{ln(\lambda)}{1-m}\frac{\lambda}{\lambda-1}\\
        \gamma &= a_{11}a_{22}-a_{12}a_{21} = ln\lambda (1-m)(\lambda + \frac{1}{\lambda -1 })
    \end{align}
    Introducing m, thus, makes $\gamma$ smaller and creates a more stable steady state.
    
    \item Since there are less parasites within the system, the average number of encounter per host should increase. We can modify $f(N_t, P_t)$ as such: \begin{align*}
        f(N_t, P_t) = e^{-(aP_t)^{c}}
    \end{align*}
    where c represents the relative degree of parasites' degree of sparsity, and $c > 1$.
    
    
\end{enumerate}
\end{homeworkProblem}
