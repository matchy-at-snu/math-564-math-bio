\begin{homeworkProblem}[20]
    Take $p_n^1$ as the number of newborns, then it should be compute by \[
    \begin{aligned}
        p_{n+1}^1 &= \sum^m_{i=1} \left(
            \begin{matrix}
                \text{number of births}\\
                \text{in age class }i\\
                \text{in last year }(n)
            \end{matrix}\right) \left(
            \begin{matrix}
                \text{number of individuals}\\
                \text{in age class }i\\
                \text{in last year }(n)
            \end{matrix}\right)\\
        &= \sum^m_{i=1} \alpha_i p_n^i
    \end{aligned}
    \]
    And for each age class $1< i < m$, it can be computed by \[
        p_{n+1}^i = \sigma_{i-1} p_{n}^{i-1}
    \]
    Since $m$ is the oldest age class, $p_n^m$ will not survive to next year.
    The system of equations can be written in the form of \[
    \begin{aligned}
        p_{n+1}^1 &= \alpha_1 p_n^1 + \alpha_2 p_n^2 + \alpha_3 p_n^3 + \cdots
        + \alpha_{m-1} p_n^{m-1} + \alpha_m p_n^m\\
        p_{n+1}^2 &= \sigma_1 p_n^1\\
        p_{n+1}^3 &= \phantom{\sigma_1 p_n^1 + }\ \sigma_2 p_n^2\\
        \vdots &\phantom{=}\\
        p_{n+1}^{m} &=
        \phantom{\alpha_1 p_n^1 + \alpha_2 p_n^2 + \alpha_3 p_n^3 +\cdots\cdots}
        \sigma_{m-1} p_n^{m-1}
    \end{aligned}
    \]
    It's clear that it can be written as the matrix form:
    \[
        \mathbf{P}_{n+1} = \mathbf{A} \mathbf{P}_n
    \]
    where \[
        \mathbf{P}_k = \left(\begin{matrix}
            p_k^1 \\ p_k^2 \\ \vdots \\ p_k^m
        \end{matrix}\right), \qquad
        \mathbf{A} = \left(\begin{matrix}
            \alpha_1 & \alpha_2 & \cdots & \alpha_{m-1} & \alpha_m  \\
            \sigma_1 & 0        & \cdots & 0            & 0         \\
            0        & \sigma_2 & \cdots & 0            & 0         \\
            \vdots   & \vdots   & \ddots & \vdots       & \vdots    \\
            0        & 0        & \cdots & \sigma_{m-1} & 0
        \end{matrix}\right)
    \]
    \end{homeworkProblem}