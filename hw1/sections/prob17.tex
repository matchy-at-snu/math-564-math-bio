\begin{homeworkProblem}[17][Annual plant propagation]
    \begin{enumerate}
        % Solution 17(a)
        \item The model for annual plants was condensed into a single equation (15)
        for $p_n$, the number of plants. Show that it can also be written as a
        single equation in $S_n^1$.
    
        \textit{Solution}: We have \begin{align}
            &p_n = \alpha s_n^1 + \beta s_n^2 \label{eq:pn} \\
            &\overline{s_n^1} = (1-\alpha)s_n^1 \label{eq:no_germ}\\
            &\overline{s_n^2} = (1-\alpha)s_n^2\\
            &s_n^0=\gamma p_n \label{eq:new_seeds}\\
            &s_{n+1}^1 = \sigma s_n^0 \label{eq:next_gen_seeds_1}\\
            &s_{n+1}^2 = \sigma \overline{s_n^1} \label{eq:next_gen_seeds_2}
        \end{align}
    
        Make several substitutions:
        \begin{align}
            &\eqref{eq:new_seeds} \rightarrow \eqref{eq:next_gen_seeds_1}
            & & S_{n+1}^1 = \sigma \gamma p_n \label{eq:sn1_next} \\
            % -------------------------------------------------------------------- %
            &\eqref{eq:no_germ} \rightarrow \eqref{eq:next_gen_seeds_2}
            & & S_{n}^2 = \sigma (1-\alpha) S_{n-1}^1 \label{eq:sn2}\\
            % -------------------------------------------------------------------- %
            &\eqref{eq:pn} \rightarrow \eqref{eq:sn1_next}
            & & S_{n+1}^1 = \sigma \gamma (\alpha s_n^1 + \beta s_n^2)
            \label{eq:res}\\
            % -------------------------------------------------------------------- %
            &\eqref{eq:sn2} \rightarrow \eqref{eq:res}
            & & S_{n+1}^1 = \sigma \gamma (\alpha s_n^1 +
            \beta \sigma (1-\alpha) S_{n-1}^1) \label{eq:sol}
        \end{align}
        And equation \eqref{eq:sol} should be the answer.
        % Solution 17(b)
        \item Seeds produced this year (year $n$) which survived the winter and will
        germinate next year (year $(n+1)$)
    
        % Solution 17(c)
        \item We know from the book that \[
            \lambda_{1,2} = \frac{\sigma \gamma \alpha}{2} (1 \pm \sqrt{1 + \delta})
        \] where \[
            \delta = \frac{4}{\gamma}\frac{\beta}{\alpha}
                    \left( \frac{1}{\alpha} - 1\right)
        \]
        We know that $\delta$ is a positive quantity since $\alpha < 1$. So we have
        a positive eigenvalue and a negative eigenvalue.
    
        If we want the population to increase in size, we should have the positive
        eigenvalue $\lambda_1 > 1$.
    
        Plug in $\alpha = \beta = 0.001$ and $\sigma = 1$, we'll have \[
        \begin{aligned}
            \delta &= \frac{4}{\gamma}(\frac{1}{0.001} - 1) \approx \frac{4000}{\gamma}\\
            \lambda_1 &= \frac{0.001\gamma}{2} (1 + \sqrt{1 + \delta})
        \end{aligned}
        \quad\rightarrow\quad
        \lambda_1 = \frac{\gamma}{2000}(1 + \sqrt{1+ \frac{4000}{\gamma}}) > 1
        \]
        Solving the inequity gives \[
            \gamma > 500
        \]
    
        % Solution 17(d)
        \item In case (1), the parameters are: \[
            \alpha = 0.5, \quad \beta = 0.25, \quad \gamma = 2.0, \quad \sigma = 0.8,
        \] we can compute that \[
        \begin{aligned}
            \delta &= \frac{4}{\gamma}\frac{\beta}{\alpha}
            \left( \frac{1}{\alpha} - 1\right)
            = 1\\
            \lambda_1 &= \frac{\sigma \gamma \alpha}{2} (1 + \sqrt{1 + \delta})
            = 0.4 (1 + \sqrt{2})\\
            &\approx 0.97 < 1
        \end{aligned}
        \]
        So the population decreases.
    
        An in case (2), the parameters are: \[
            \alpha = 0.6, \quad \beta = 0.3, \quad \gamma = 2.0, \quad \sigma = 0.8,
        \] we can compute that \[
        \begin{aligned}
            \delta &= \frac{4}{\gamma}\frac{\beta}{\alpha}
            \left( \frac{1}{\alpha} - 1\right)
            = \frac{2}{3}\\
            \lambda_1 &= \frac{\sigma \gamma \alpha}{2} (1 + \sqrt{1 + \delta})
            = 0.48 (1 + \sqrt{5/3})\\
            &\approx 1.10 >1
        \end{aligned}
        \]
        So the population increases.
    
        % Solution 17(3)
        \item Consider the positive eigenvalue as \[
            \lambda_1 = \frac{a + \sqrt{a^2 + 4b}}{2}
        \] where \[
            a = \alpha \sigma \gamma, \quad b = \beta \sigma^2 (1-\alpha)\gamma
        \]
        If we want $\lambda_1 > 1$, then \[
        \begin{aligned}
            \frac{a + \sqrt{a^2 + 4b}}{2} &> 1\\
            \sqrt{a^2 + 4b} &> 2-a\\
            a^2 + 4b  &> a^2 - 4a + 4\\
            a + b &> 1
        \end{aligned}
        \] (Note: If $(2-a) < 0$ then the inequity will always be trivially true,
        thus it's not of too much interest. We only consider $(2-a) > 0$ here.)
        Substitute back $a = \alpha \sigma \gamma$ and
        $b = \beta \sigma^2 (1-\alpha)\gamma$, we'll have \[
            \gamma > \frac{1}{\alpha \sigma + \beta \sigma^2 (1-\alpha)}
        \]
    \end{enumerate}
    \end{homeworkProblem}