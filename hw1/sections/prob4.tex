\begin{homeworkProblem}
    In Section 1.4 we determined that there are two values $\lambda_1$ and
    $\lambda_2$ and two vectors $\left(\begin{matrix}A_1\\B_1\end{matrix}\right)$
    and $\left(\begin{matrix}A_2\\B_2\end{matrix}\right)$ called
    \textit{eigenvectors} that satisfy equation (29).
    
    \begin{enumerate}
        \item Show that this equation can be written in matrix form as
        \[
        \lambda\left(\begin{matrix}
            A\\B
        \end{matrix}\right) = \mathbf{M} \left(\begin{matrix}
            A\\B
        \end{matrix}\right)
        \]
        where $\mathbf{M}$ is given by equation (27c).
        \item Show that one way of expressing the eigenvectors in terms of $a_{ij}$
        is: \[
            \left(\begin{matrix}
                A_i\\B_i
            \end{matrix}\right) = \left(\begin{matrix}
                1\\ \frac{\lambda_i - a_{11}}{a_{12}}
            \end{matrix}\right)
        \]
        for $a_{12} \neq 0$.
        \item Show that eigenvectors are defined only up to a multiplicative
        constant; i.e., if $\mathbf{v}$ is an eigenvector corresponding to the
        eigenvalue $\lambda$, then $\alpha\mathbf{v}$ is also an eigenvector
        corresponding to $\lambda$ for all real numbers $\alpha$.
    \end{enumerate}
    
    \segline
    
    \solution
    \begin{enumerate}
        \item Equation (29) in the book is \[
            \begin{aligned}
                0 &= A(a_{11} - \lambda) + B(a_{12})\\
                0 &= A(a_{21}) + B(a_{22} - \lambda)
            \end{aligned}
        \]
        Re-arranging the terms with $\lambda$ to the left-hand side gives\[
            \begin{aligned}
                \lambda A &= a_{11} A + a_{12} B\\
                \lambda B &= a_{21} A + a_{22} B
            \end{aligned}
        \]
        which is equivalent of \[
            \lambda \left(\begin{matrix}
                A\\B
            \end{matrix}\right) = \left(\begin{matrix}
                a_{11} & a_{12}\\
                a_{21} & a_{22}
            \end{matrix}\right) \left(\begin{matrix}
                A\\B
            \end{matrix}\right) = \mathbf{M} \left(\begin{matrix}
                A\\B
            \end{matrix}\right)
        \]
        \item \textit{Proof}. Suppose the eigenvalue is $\lambda_i$ and the
        eigenvector corresponding to it is $\mathbf{v}_i = \left(\begin{matrix}
            A_i\\B_i
        \end{matrix}\right)$. Then we know by the definition of eigenvector that \[
            \left(\begin{matrix}
                a_{11} & a_{12}\\
                a_{21} & a_{22}
            \end{matrix}\right) \mathbf{v}_i = \lambda_i \mathbf{v}_i.
        \]
        Suppose $\mathbf{v}_i = \left(\begin{matrix}
            1\\
            \frac{\lambda_i - a_{11}}{a_{12}}
        \end{matrix}\right)$, then\[
            \left(\begin{matrix}
                a_{11} & a_{12}\\
                a_{21} & a_{22}
            \end{matrix}\right) \mathbf{v}_i =
            \left(\begin{matrix}
                a_{11} & a_{12}\\
                a_{21} & a_{22}
            \end{matrix}\right)
            \left(\begin{matrix}
                1\\ \frac{\lambda_i - a_{11}}{a_{12}}
            \end{matrix}\right) =
            \left(\begin{matrix}
                \lambda_i \\
                a_{21} + \frac{a_{22}(\lambda_i - a_{11})}{a_{12}}
            \end{matrix}\right)
        \]
        And \[
            \lambda_i \mathbf{v}_i = \left(\begin{matrix}
                \lambda_i \\
                \frac{\lambda_i^2 - a_{11}\lambda_i}{a_{12}}
            \end{matrix}\right)
        \]
        Suppose \[
            \left(\begin{matrix}
                \lambda_i \\
                a_{21} + \frac{a_{22}(\lambda_i - a_{11})}{a_{12}}
            \end{matrix}\right) = \left(\begin{matrix}
                \lambda_i \\
                \frac{\lambda_i^2 - a_{11}\lambda_i}{a_{12}}
            \end{matrix}\right)
        \]
        since we have $a_{12} \neq 0$, there will be \[
            \lambda_i^2 -a_{11}\lambda_i =
            a_{21}a_{12} + a{22} \lambda_i - a+{11}a_{22}
        \]
        which gives \[
            \lambda_i = \frac{\beta \pm \sqrt{\beta^2 - 4\gamma}}{2}
        \]
        where \[
            \beta = a_{11} + a_{22}, \quad \gamma = (a_{11}a_{22} - a_{21}a_{12}).
        \]
        And that's exactly the value of $\lambda_i$.
    
        Thus we have shown that $\mathbf{v}_i = \left(\begin{matrix}
            1\\ \frac{\lambda_i - a_{11}}{a_{12}}
        \end{matrix}\right)$ satisfies
        $\mathbf{M}\mathbf{v}_i = \lambda_i \mathbf{v}_i$,
        i.e. it is indeed one way of expressing the eigenvector.
        \begin{flushright}
            $\qed$
        \end{flushright}
    
        \item \textit{Proof}. Suppose that $\mathbf{v}$ is the eigenvector
        corresponding to eigenvalue
        $\lambda$, then it is known that \[
            \mathbf{M}\mathbf{v} = \lambda\mathbf{v}
        \]
        For $\alpha\mathbf{v}$, we know from the properties of scalar multiplication
        \[
            \mathbf{M} \left(\alpha \mathbf{v}\right) =
            \alpha \left(\mathbf{M} \mathbf{v}\right)
        \]
        and \[
            \lambda(\alpha\mathbf{v}) = \alpha(\lambda \mathbf{v})
        \]
        It's clear that the two quantities are equal. Thus, by the definition of
        eigenvectors, $\alpha \mathbf{v}$ is also an eigenvector.
        \begin{flushright}
            $\qed$
        \end{flushright}
    \end{enumerate}
    \end{homeworkProblem}