\begin{homeworkProblem}[16][Red blood cell production]
    \begin{enumerate}
    % Solution 16(a)
    \item We have \begin{align}
        R_{n+1} &= (1-f) R_n + M_n, \label{eq:RBC_in_circulation}\\
        M_{n+1} &= \gamma f R_n \label{eq:RBC_produced_nextgen}
    \end{align}
    Equation \eqref{eq:RBC_produced_nextgen} can be re-written into \begin{equation}
        M_{n} = \gamma f R_{n-1} \label{eq:RBC_produced}
    \end{equation}
    Substitute \eqref{eq:RBC_produced} into \eqref{eq:RBC_in_circulation}, we'll get
    \[
        R_{n+1} = (1-f) R_n + \gamma f R_{n-1}
    \]
    
    % Solution 16(b)
    \item From (a), it's easy to see that the characteristic function is \[
        \lambda^2 - (1-f) \lambda - \gamma f
    \]
    Hence, the eigenvalues are indeed given by \[
        \lambda_{1,2} = \frac{(1-f) \pm \sqrt{(1-f)^2} + 4\gamma f}{2}
    \]
    Since $f$ is definitely a fraction value between $0$ and $1$ (it's horrible to
    think that your spleen removes 100\% of your RBCs in circulation), and $\gamma$
    is larger than $0$, so that there will be RBC produced.
    
    Observe the boundary stated above, which makes the model biologically reasonable
    , we know \[
        \sqrt{(1-f)^2 + 4\gamma f} > \sqrt{(1-f)^2} = (1-f)
    \]
    Therefore, we should have a positive and a negative eigenvalue, i.e.,
    $\lambda_1 > 0$ and $\lambda_2 < 0$.
    
    % Solution 16(c)
    \item Suppose the positive eigenvalue $\lambda_1 = 1$, then \[
        \begin{aligned}
        \frac{(1-f) + \sqrt{(1-f)^2} + 4\gamma f}{2} &= 1\\
        \sqrt{(1-f)^2} + 4\gamma f &= 1+f\\
        (1-f)^2 + 4\gamma f &= (1+f)^2\\
        4\gamma f &= 4f\\
        \gamma &= 1
        \end{aligned}
    \]
    
    % Solution 16(d)
    \item From (c) we know $\gamma = 1$, so \[
        \sqrt{(1-f)^2} + 4\gamma f = 1+f
    \]
    Then \[
        \lambda_2 = -2f/2 = -f
        \]
    And the solution \[
        R_n = A \lambda_1^n + B \lambda_2^n = A + B(-f)^n
    \]
    The solution will oscillate around constant $A$, and since $f$ is a fraction the
    oscillation amplitude will decrease as $n$ increases. This means that the RBC
    number in circulation will eventually reach an equilibrium.
    \end{enumerate}
    \end{homeworkProblem}