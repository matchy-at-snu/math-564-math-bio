\begin{homeworkProblem}[18][Blood $CO_2$ and ventilation]
    \begin{enumerate}
    % Solution 18(a)
    \item We know from the book that \begin{align}
        C_{n+1} &= C_{n} - \mathcal{L}(V_n, C_n) + m, \label{CO2_next}\\
        V_{n+1} &= \mathcal{S}(C_n). \label{ventilation_next}
    \end{align}
    From the question description we know that \[
        \mathcal{L}(V_n, C_n) = \beta V_n, \qquad V_{n+1} = \alpha C_n
    \]
    Make some simple substitution, we'll have \[
        C_{n+1} = C_{n} - \alpha \beta C_{n-1} + m
    \]
    i.e. \[
        C_{n+1} - C_n + \alpha\beta C_{n-1} = m
    \]

    % Solution 18(b)
    \item \begin{enumerate}[label=(\arabic*)]
    % Solution 18(b)(1)
    \item Suppose $C_n = m / \alpha \beta$, then \[
        C_{n+1} - C_n + \alpha \beta C_{n-1}
        = m/\alpha\beta - m/\alpha\beta + \alpha\beta \cdot \frac{m}{\alpha\beta}
        = m
    \]
    Hence, $C_n = m/\alpha \beta$ is a particular solution.
    % Solution 18(b)(2)
    \item Let $m = 0$, then it'll turn into a homogeneos problem. The characteristic
    equation would be \[
        \lambda^2 - \lambda + \alpha\beta = 0
    \]
    with eigenvalues \[
        \lambda_{1,2} = \frac{1 \pm \sqrt{1 - 4\alpha\beta}}{2},
    \]
    which is the \textit{complementary/homogeneous solution}.

    Combined the complementary solution with the particular solution, the general
    solution would be \[
        C_{n} = \frac{m}{\alpha\beta} +
                C_1 \left(\frac{1 + \sqrt{1 - 4\alpha\beta}}{2}\right)^n +
                C_2 \left(\frac{1 - \sqrt{1 - 4\alpha\beta}}{2}\right)^n
    \]
    \end{enumerate}% end 18(b)

    % Solution 18(c)
    \item \begin{enumerate}[label=(\arabic*)]
    % Solution 18(c)(1)
    \item Assume that $4\alpha\beta < 1$, then $\alpha\beta < 1/4$. This implies \[
        \mathcal{L}(V_n, C_n) = \beta V_n = \alpha \beta C_{n-1} < \frac{C_{n-1}}{4}
    \]
    i.e., the amount of $CO_2$ loss / breathed out will not be larger than
    $\frac{1}{4}$ of the blood $CO_2$ amount.

    And, since $4 \alpha\beta < 1$, then $ 0 < \sqrt{1 - 4\alpha\beta} < 1$. Let
    $\sqrt{1 - 4\alpha\beta} = \delta$, then \[
    \begin{aligned}
        \lambda_1 &= \frac{1+\delta}{2} \in (\frac{1}{2}, 1)\\
        \lambda_2 &= \frac{1-\delta}{2} \in (0, \frac{1}{2})
    \end{aligned}
    \]
    Therefore, the absolute value of each eigenvalue $|\lambda_i| < 1$, implying \[
        \lim_{n \to \infty} (C_{n}) = \frac{m}{\alpha\beta}
    \] since the power of a fraction will diminish towards $0$.
    Under this scenario, a steady state will eventually be established regardless of
     the initial conditions. And the steady ventilation rate would be \[
        V_{n} = \alpha \frac{m}{\alpha\beta} = \frac{m}{\beta}
    \]

    %Solution 18(c)(2)
    \item Assume that $4\alpha\beta > 1$, then $1 - 4\alpha\beta < 0$, which means
    that the characteristic equation would have complex eigenvalues. The conjugated
    complex eigenvalues $\lambda = a \pm bi$ have \[
        a = \frac{1}{2}, \quad b = \frac{\sqrt{4\alpha\beta - 1}}{2}
    \]
    i.e. \[
        r = \sqrt{a^2 + b^2} = \sqrt{\alpha\beta}, \quad \theta = \tan^{-1}(b/a)
        = \tan^{-1}(\sqrt{4\alpha\beta - 1})
    \]
    Then \[
        C_n = r^n (C_1 \cos(n\theta) + C_2 \sin(n\theta))
        = (\sqrt{\alpha\beta})^n \sin(n\theta + \phi)
    \]
    where $\tan \phi = C_1/C_2$. It is clear that the solution will oscillate at
    frequency $f = \theta/2\pi$

    If $\alpha\beta \geq 1$, then the oscillation will increase in magnitude. When
    $\alpha\beta = 1$, the oscillation frequency would be $f = \tan^{-1}(\sqrt{3}) =
    \frac{\pi}{3}$.

    This result could correspond to Cheyne-Stokes respiration, which is a disorder
    characterized by recurrent oscillation between apnea and hyperpnea (May, 1978
    and Naughton, 1998).
    \end{enumerate}% end 18(c)

    % Solution 18(d)
    \item For now the ventilation rate is proportional to blood $CO_2$
    concentration. In real biological systems, body will likely adjust the rate of
    breath according to $C_n$, just like the relationship between air friction and
    velocity. It's suitable to suppose that \[
        \mathcal{S}(C_n) = \alpha C_n^2
    \]
    Then the equation will turn into \[
        \begin{aligned}
        &C_{n+1} = C_n - \alpha \beta C_{n-1}^2 + m
        &C_{n+1} - C_n + \alpha \beta C_{n-1}^2 = m
        \end{aligned}
    \]
    Suppose $C_{n+1}=C_n$, then we can derive a steady-state where \[
        C_n = \sqrt{\frac{m}{\alpha\beta}}
    \]
    It would also be interesting to think about the equilibrium shifting between
    bicarbonates and $CO_2$ and the formation of carbaminohemoglobin($HbCO_2$).
    But deriving a nice math equation with respect to these factors is out of our
    ability.

    \end{enumerate}
    \end{homeworkProblem}